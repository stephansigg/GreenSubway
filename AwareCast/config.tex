\usepackage{graphicx} % Graphikunterstuetzung
\usepackage{wrapfig} % um Bilder vom Text umfliessen zu lassen
\usepackage{subfigure} % fuer mehrere Grafiken ((a)-, (b)-Graphiken) in einer figure-Umgebung

%-- Package hyperref ------------------------------------------------------------------------------
\usepackage[
	plainpages=false, %Gibt an auf welcher Seite die pdf-Darstellung beginnt.
	pdfpagelabels,
	pdftex=true,
	breaklinks=true, %/false: Gibt an, ob Links umgebrochen werden duerfen.
	%linktocpage=true/false: im Inhaltsverzeichnis sind nur die Seitenzahlen links, nicht der Text
	colorlinks=true, %/false: Links werden eingefaerbt (Farben werden mit linkcolor, anchorcolor ... festgelegt)
	linkcolor=black, %Farbe des verlinkten Textes, Dokument-interne Links
	citecolor=black, %Farbe des verlinkten Textes, Links zum Literaturverzeichnis
	filecolor=black, %Farbe des verlinkten Textes, Links auf lokale Dateien
	urlcolor=black, %Farbe des verlinkten Textes, externe URLs
	%frenchlinks=true/false: Links werden als smallcaps, anstatt farbig dargestellt.
	menucolor=black
]{hyperref}

\hypersetup{
	pdftitle={GreenSubway},
	pdfauthor={Andreas Jahn, Stephan Sigg},
	pdfsubject={GreenSubway},
	pdfkeywords={GreenSubway},
	%pdfpagelayout={TwoColumnRight}
	%bookmarksnumbered=true,
	%bookmarksopen=true,
	%bookmarksopenlevel=1,
	%pdfpagemode=None % None, UseOutline, UseThumbs, FullScreen
}
\usepackage{balance}  % to better equalize the last page

%-- penalties --------------------------------------------------------------------------------
\clubpenalty = 10000 % Schusterjunge verhindern. Wird beim Seitenumbruch vergeben, falls die erste Zeile eines Absatzes allein auf der vorangehenden Seite verbleibt. default=150
\widowpenalty = 10000 % Wird beim Seitenumbruch aufgerechnet, falls die letzte Zeile gerade noch auf die nächste Seite umgebrochen wird. default=150
\displaywidowpenalty = 10000 % Hurenkinder verhindern. Wird während des Seitenumbruchs für einen Seitenwechsel angerechnet, falls die letzte Zeile eines Absatzes gerade noch auf die nächste Seite kommt, und zwar in dem Spezialfall, dass direkt danach eine hervorgehobene Formel folgt.
\finalhyphendemerits = 10000 % Wird für die Trennung in der vorletzten Zeile eines Absatzes aufgerechnet. default=5000
\linepenalty = 100 % Grundlast, die für jede Zeile berechnet wird. Wird dieser Wert erhöht, so versucht das Programm, die Zeilenanzahl für einen Absatz möglichst klein zu halten.
%\hyphenpenalty = 500 % Wird für eine einfache Trennung beim Absatzumbruch aufgerechnet.
\exhyphenpenalty = 500 % Wird für einen Zeilenwechsel nach einem expliziten Trennstrich aufgerechnet.
%\binoppenalty = 700 % Mathematiksatz: Gibt es für das Trennen einer Formel im text-style nach einem binären Operator.
%\relpenalty = 500 % Wird beim Trennen einer mathematischen Formel im text-style (nach einer Relation) aufgerechnet.
\brokenpenalty = 100 % Wird während des Seitenumbruchs für einen Seitenwechsel angerechnet, bei dem die letzte Zeile einer Seite mit einer Trennung endet.
\doublehyphendemerits = 10000 % Für zwei aufeinanderfolgende Zeilen, wobei in beiden getrennt wird.
\adjdemerits = 10000 % Wird beim Absatzumbruch für zwei »visuell inkompatible« Zeilen aufgerechnet. Dies sind zwei aufeinanderfolgende Zeilen, bei denen die eine mit extra viel Leerraum durchgeschossen wurde und die andere mit wenig.
\interdisplaylinepenalty = 10000 % Wird für das Umbrechen einer mit \displaylines entstandenen Formelfolge über eine Seitengrenze hinweg berechnet.
% \interfootnotelinepenalty = 100 % Wird für das Umbrechen einer Fußnote über Seitengrenzen hinweg berechnet. Der Befehl \footnote setzt die \interlinepenalty auf den angegebenen Wert.
%\interlinepenalty = 5000 %Für den Umbruch eines Absatzes über eine Seitengrenze.
\predisplaypenalty = 5000 %Für einen Seitenwechsel direkt vor einer Formel im display-style.
\postdisplaypenalty = 5000 %Für einen Seitenwechsel direkt nach einer Formel im display-style.