%-- Package hyperref ------------------------------------------------------------------------------
\usepackage[
	plainpages=false, %Gibt an auf welcher Seite die pdf-Darstellung beginnt.
	pdfpagelabels,
	pdftex=true,
	breaklinks=true, %/false: Gibt an, ob Links umgebrochen werden duerfen.
	%linktocpage=true/false: im Inhaltsverzeichnis sind nur die Seitenzahlen links, nicht der Text
	colorlinks=true, %/false: Links werden eingefaerbt (Farben werden mit linkcolor, anchorcolor ... festgelegt)
	linkcolor=black, %Farbe des verlinkten Textes, Dokument-interne Links
	citecolor=black, %Farbe des verlinkten Textes, Links zum Literaturverzeichnis
	filecolor=black, %Farbe des verlinkten Textes, Links auf lokale Dateien
	urlcolor=black, %Farbe des verlinkten Textes, externe URLs
	%frenchlinks=true/false: Links werden als smallcaps, anstatt farbig dargestellt.
	menucolor=black
]{hyperref}

\hypersetup{
	pdftitle={GreenSubway},
	pdfauthor={Andreas Jahn, Stephan Sigg},
	pdfsubject={GreenSubway},
	pdfkeywords={GreenSubway},
	%pdfpagelayout={TwoColumnRight}
	%bookmarksnumbered=true,
	%bookmarksopen=true,
	%bookmarksopenlevel=1,
	%pdfpagemode=None % None, UseOutline, UseThumbs, FullScreen
}
\usepackage{balance}  % to better equalize the last page