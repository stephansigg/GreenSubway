\section{Introduction}
\label{sec:introduction}

Underground transportation systems are big energy consumers and have significant impacts on energy consumptions at a regional scale~\cite{anderson_maximizing_2009}. Approximately 30\% of the required energy is needed for operating the metro stations and surroundings, such as ventilation, vertical transportation and lighting~\cite{TMB}.

To realize energy saving in this area already an intelligent control system for metro stations was developed. The control system is adaptive on the basis of environmental factor forecasts and occupancy flow patterns~\cite{guo_intelligent_2013}.

Changing the parameter doesn't have an immediate effect. Therefore it would the number of passenger needs to be predicted.

This paper focuses the prediction of number of passenger in the station.



\subsection{Data Acquisition}
\label{subsec:dataAcquisition}

Metro Station Passeig de Gracia Line 3 (PdGL3).

Data are gathered via CCTV. The images are processed. Out of each image the number of passenger is gathered. In this way a database was filled which contains for each data set the time, location and value.