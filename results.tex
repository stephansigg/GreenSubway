\section{Results}
\label{sec:results}

This section focuses on the evaluation and the results. First the evaluation of the recorded sensor data is shown. Subsequently the results are presented and discussed.

\subsection{Evaluation}
\label{subsec:evaluation}

The evaluation analyses the prediction occupancy and provides a measurement in order to depict the prediction performance. The evaluation in detail is depicted in the following.

\subsubsection{Performance measurement}
\label{subsubsec:performanceMeasurement}

In order to understand how good the predicted value and the actual value match, a performance measurement is needed. A standard measurement is the accuracy. The accuracy describes the performance of the system in a percent value. An accuracy value of 100~\% represents a perfect prediction, while 0~\% represents a poor prediction. In case of the Usermodel an easy way of calculation could be, a division of the lower number of passenger by the higher number of passenger value as depicted in equation~\ref{eq:accuracy}.

\begin{equation}
accuracy = \frac{lower~value}{heigher~value}
\label{eq:accuracy}
\end{equation}

Following this equation the accuracy for a predicted value of two and an actual value of one is calculated to 50~\% (equation~\ref{eq:accuracyExample}).

\begin{equation}
accuracy = \frac{1}{2} = 0.5 = 50~\%
\label{eq:accuracyExample}
\end{equation}

In this example the accuracy is 50~\% even though the predicted value is close to the actual value. Predicted and actual value differs only by one. A difference of one person does not have big impact on the controller in order to satisfy the restrictions. Therefore the accuracy does not fulfil the needs as a performance measure.

Instead the absolute difference between the predicted number of passenger as well as the actual number of passenger seems to provide a meaningful measurement. A measurement-value of zero~(0) represents a perfect prediction, while the higher the worse the prediction. In this deliverable absolute difference between predicted number of passenger and the actual number of passenger is used as the performance measure for the detection accuracy of the number of passenger prediction.
Staying in the mentioned example the accuracy is calculated to one (Equation~\ref{eq:diffExample}).

\begin{equation}
accuracy = |1-2| = 1
\label{eq:diffExample}
\end{equation}

One is already close to zero and allows therefore the conclusion of a good prediction. 


\subsection{Results}
\label{subsec:results}

This section depicts the prediction results and is aimed to figure out:

\begin{enumerate}
  \item the overall prediction accuracy,
  \item the best performing penalty setup,
  \item the best performing history and observation length, and
  \item the average prediction duration.
\end{enumerate}
